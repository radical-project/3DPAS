\documentclass[12pt]{article}

\usepackage{graphicx} 
\usepackage{amssymb}
\usepackage{epstopdf}
\usepackage{array}
\usepackage{fancyhdr}
\usepackage{url}
\usepackage{hyperref}
\usepackage{pdfsync}
\pagestyle{fancy}
\usepackage{mdwlist}
\fancyhead{}
\renewcommand{\headrulewidth}{0pt}
\usepackage{hyperref}

% \parskip=0.075in
% \topmargin=-1.25in
% \oddsidemargin=1.0in
% \evensidemargin=1.0in
% \textheight=9.0in
% \textwidth=6.5in
\usepackage{fullpage}
%\usepackage[top=tlength, bottom=blength, left=llength, right=rlength]{geometry}
%http://en.wikibooks.org/wiki/LaTeX/Page_Layout
%\usepackage[margin=1in, paperwidth=5.5in, paperheight=8.5in]{geometry}

\usepackage{fancyhdr}
\setlength{\headheight}{16.0pt}
\pagestyle{fancy}

%\pagestyle{fancyplain}
 
% \fancyhf{} \chead{\fancyplain{}{{\bf {\it Jha: Cyberinfrastructure
%         Challenges for Dynamic Data Scenarios in X-Scale CDES}}}}

\headheight = 0pt
\headsep = 25pt

\fancyhf{} \fancyhead[OC]{\bf {\it \footnotesize{Jha:
      Cyberinfrastructure Challenges for Dynamic-Data in Extreme-Scale
      CDES}}}


% in Extreme-Scale Computational and Data-Enabled Science}}}
%\rhead{\fancyplain{}{\footnotesize{Computational and Data-Enabled Science}}}
%\rfoot{\fancyplain{}{\thepage}}

\begin{document}

\newcommand{\squishlist}{
  \begin{list}{$\bullet$} { \setlength{\itemsep}{0pt}
      \setlength{\parsep}{3pt} \setlength{\topsep}{3pt}
      \setlength{\partopsep}{0pt} \setlength{\leftmargin}{1.5em}
      \setlength{\labelwidth}{1em} \setlength{\labelsep}{0.5em} } }

  \newcommand{\squishend}{
\end{list}  }
\vspace{-1.0in}

\noindent {\bf Shantenu Jha} is currently the Director of Cyberinfrastructure
Development at the CCT at LSU. He will be a tenure-track Assistant
Professor starting in Spring 2011. Jha's interest are at the triple
point of High-Performance \& Distributed Computing, Computation and
Data-enabled Science (CDES), and Cyberinfrastructure Research \& Development.
He is the lead of the SAGA project (\url{http://www.saga.cct.lsu});
SAGA provides a high-level API for distributed functionality and the
building blocks to develop distributed application (primarily
compute-intensive) and distributed programming systems. Jha is also
the PI for the UK e-Science Institute Research Themes on Distributed
Programming Abstractions (DPA) and the current Distributed Dynamic
Data-Intensive Programming Abstractions and Systems (3DPAS).  He is
the PI of an active NSF-OCI award to establish and facilitate US
collaborations with the UK e-Science Institute 3DPAS Research theme on
Distributed Dynamic Data. Jha is also the PI of the ~\$2M NSF-EPSCOR
Cybertools Project (http://cybertools.loni.org/). Jha is a
contributing co-editor of the chapter on ``Cyberinfrastructure
Requirements and Challengens'' of AFOSR/NSF Co-Sponsored Report on
Dynamic Data Driven Applications Systems~\cite{dddas}.

\noindent  {\bf Vision:\footnote{Much of the underlying theoretical context and
    motivating examples, have been developed in collaboration with
    Jose Fortes (Florida), Daniel Katz (UC/ANL), Manish Parashar
    (Rutgers) and Jon Weissman (Minnesota).  Specifically, Jha has
    been examining the influence of cyberinfrastructure on large-scale
    distributed scientific
    applications\cite{dpa,3dpas,dpa-survey-paper} in conjunction with
    Katz, Weissman and Parashar. } }  % Many problems
% at the forefront of science, engineering, medicine, and the social
% sciences, are increasingly complex and interdisciplinary due to the
% plethora of data sources and computational methods available today.
A common feature across many of scientific and engineering problem
domains is the amount and diversity of data and computation that must
be integrated to yield insights. % For example, large volumes of data
% are being increasingly generated by/from distributed sensors,
% scientific instruments, and simulations.
An increasingly important type of pervasive scientific and engineering
applications are data-driven applications. Such applications involve
computational activities triggered as a consequence of independent
data creation. For example, in large-scale applications such as LEAD,
data streams from the sensors drive the execution of the
computations. Often in response to a predicted or phenomenologically
interesting event, the data source and streams themselves need to be
adapted, e.g., the sampling rates or resolutions are
changed. Additional elements of dynamic compute and data adaptation
arise from spatio-temporal variations in data generation (sensors). In
many dynamic applications, processing of data often needs to take
place in-flight (e.g, via streaming) to meet boundary conditions,
and/or data volume reduction may be required. For example, in the
International Thermonuclear Experimental Reactor (ITER) application,
data has to be transformed while it is being streamed. Understanding
how to support dynamic computations is thus a fundamental, often
critically missing, element in pervasive data-intensive computing.

For a large class of dynamic data applications, data must be
integrated with extreme-scale computations and analytics as part of
end-to-end application workflows. Some specific motivations for the
need for such integration of data with scalable computing capabilities
and techniques are: (1) Exascale simulations will lead to exascale
datasets; exabytes of data will require exaflops of computing
capability to analyze and mine; (2) Large data-set sizes and the need
to operate on them in a time-bounded manner (e.g., real-time video
data-streams for image detection); (3) Large number of data-souces
arising from large geographical coverage (e.g., sensors used in
Hurricane Modelling); (4) Adaptive simulation models coupled with
disparate, varying and refined in realtime data models/sources
(e.g. Gulf of Mexico Oil Spill dispersion trajectory prediction).  We
note that the challenges of managing exaflops of computing driven by
large amounts of data are different from those of exaflops of
computing that generate exabytes of data. The ultimate goal is to
address both sets of challenges as a collective whole, but the path to
that goal will require detailed and specific capabilties for each
individual scenario. For the purposes of the PeCS workshop, we will
pursue the vision, capabilities and challenges inherent in the former,
i.e., coupling extreme scales of computing driven by large amounts of
data.

% Petabyte datasets are already common and terascale to exascale
% datasets are projected in many science fields;

\noindent {\it Challenges in achieving the Vision:} The landscape of
developing and deploying dynamic data-driven applications is dominated
by the existence of partial and often non-overlapping approaches (for
example incomplete solutions for local data versus distributed data,
analysis on static data versus dynamic data, management techniques for
small data versus big data). This is true along at least three
dimensions of cyberinfrastructure -- programming sytems (PS),
middleware and the tools \& services. This limits the capabilities
offered to scientists, effective and efficient application development
and thus the impact of extreme-scale CDES.

\noindent {\it How our vision will have impact: } As we move towards
scientific problems of ever-greater complexity and significance, which
require that we address the next-scale in compute and data, it is
imperative to understand and to design integrated cyberinfrastructure
-- programming systems, middleware, and tools and services -- for
application scenarios that have a critical need to couple dynamic and
distributed data into the analytics control-loop. For example, in
climate science, the next IPCC is expected to generate hundreds of
petabytes of data, which in turn is based upon petabytes of input data
arising from sensors. 

\noindent Specific issues that our future research will contribute to:

% in genomics,
% the Joint Genome Institute alone will have about 1 PB of data this
% year and this is doubling each year.


% For dynamic data-intensive scientific applications at extreme-scale \&
% performance. In many of these application scenarios, there is a
% critical need to couple dynamic and distributed data into the
% analytics control-loop.

% for dynamic data-intensive scientific applications at
% extreme-scale \& performance. In many of these application scenarios,
% there is a critical need to couple dynamic and distributed data into
% the analytics control-loop. 

\begin{itemize*}\vspace{-0.075in}
\item Analyzing the landscape of applications \&
  cyberinfrastructure %and the programming system 
  for distributed and dynamic-data scenarios

\item Development of cyberinfrastructure to support
  dynamic-data as a first-class concern % of distributed applications.

\item Design and implementation of %general-purpose %and pervasive
  cyberinfrastructure that supports dynamic data, but which adheres
  to open-standards,
  supports many applications types across multiple spatio-temporal scales,
  and can be deployed on a range of production
  distributed systems across scales.
 
%   Conversely,
%   the implementation and 
%   Conversely,
%   the implementation and deployment on a range of production
%   distributed systems across scales.

\item Role of autonomic decision making in dynamic-data \&
  distributed compute-data co-location challenges. Design \&
  development of middleware to
  support these capabilities 
\end{itemize*}

% Our research agenda will address the following objectives of the PeCS
% workshop:

% \begin{itemize*}\vspace{-0.075in}
% \item Identify software challenges, including middleware and operating
%   systems, for pervasive computing and associated applications;
% \item Explore the nexus between scalability and application
%   characteristics and context with the goal of identifying fundamental
%   insights, models and methods;
% \item Explore new interfaces and modes of interactions between people
%   and pervasive computing devices, applications or environments; 
% % \item In conjunction with the 3DPAS Research Themes,
% %   we will contribute to the following objective: ``Identify open
% %   problems
% %   and fundamental challenges that must be addressed to enable
% %   deployment of pervasive computing systems at massive scale''.
% \end{itemize*}

% SAGA is usable with almost every major middleware globally and is
% supported on nearly ever major compute intensive production grid
% infrastructure (such as NSF TeraGrid, European EGI etc).

% High-end data-intensive applications, the programming systems and
% tools, and both the current and to-be-developed middleware that
% support them.
 
% Our efforts will build upon sustained collaboration in this domain
% over the past few years. This workshop will enable us to use the
% methodology we have developed for large-scale distributed applications
% for dynamic data-intensive applications and middleware, to explore
% opportunities to leverage existing and emerging technologies to
% address extreme-scale data challenges and identify the gaps..
 
\vspace{-0.25in}
\begin{footnotesize}
\begin{thebibliography}{9}

\bibitem{dpa} DPA: Distributed Programming Abstraction e-Science
  Institute Research Theme. \\
  \url{http://wiki.esi.ac.uk/Distributed_Programming_Abstractions}

\bibitem{3dpas} 3DPAS: Dynamic, Distributed and Data-Intensive Programming
  Abstractions and Systems.  \\ \url{http://wiki.esi.ac.uk/3DPAS}

% \bibitem{3dpas-hpdc} 3DAPAS: Dynamic, Distributed and Data-Intensive
%   Applications, Programming Abstractions and
%   Systems. \url{https://sites.google.com/site/3dapas/}

\bibitem{dddas} Chapter on Cyberinfrastructure for Dynamically
  Data-Driven Application Scenarios, Edited by Gabrielle Allen and
  Shantenu Jha, AFOSR/NSF Co-Sponsored Workshop on Dynamic Data Driven
  Applications
  Systems. \url{http://www.dddas.org/AFOSR-NSFworkshop2010.html}

\bibitem{dpa-survey-paper} Distributed Computing Practice for
  Large-Scale Science \& Engineering Applications, Under review, by
  Jha, Cole, Katz, Rana, Parashar and Weissman. \\
  \url{https://cct.lsu.edu/~sjha/dpa_publications/dpa_surveypaper.pdf}

% \bibitem Data-Intensive Research Workshop Report, edited by Atkinson,
%   Jha et al.  \url{http://dl.dropbox.com/u/3073925/DIRWS.pdf}

\end{thebibliography} 
\end{footnotesize}
\end{document}


Understanding the primary characteristics, usage scenarios \&
requirements of 3D applications, and then developing the capabilities
to support 3D applications within the same fold as compute intensive
applications will be critical.

\subsection{Old Notes from 3.2}

\note{Old classification:
\begin{itemize}
	\item distributed by design (prev. naturally distributed): the data sources
	themselves are naturally distributed, e.g. scientific instruments or sensors
	collecting observational data.
    \item replicated: the data is replicated and therefore distributed to
	multiple locations
	\item partitioned: the data is partitioned and distributed across
	multiple locations
    \note{possible replacements for the three above - problem: all applications have them!} \note{What is a data source? How many data sources are in Atlas?}
	\item coupled \alnote{alternatively aggregation?}: multiple data sources (M)
	are combined at one or more locations (N), M$>$N
   \jhanote{do we propose natural, replicated and or partitioned as the 3
   types? for coupled can be used as a qualifier for all three of the
   above.}
   \item replicated: multiple data sources (M) are expanded to
          one or more locations (N): M$ <$ N \jhanote{can this be
          merged with above replicated?}
\end{itemize}
}


\note{View point: We should probably aim to classify applications with respect to the data processing piece? How do we describe application with multiple
distributed execution units and the dataflow in between?}

\note{Is the production of the data inside or outside of the application? When
is data production infrastructure and when is it application? DK: Floating
sensors are an instrument that happened to physically distributed. Are sensor as a whole designed to function at all}
\note{What kind of control do you have over the data source?}


\note{federated data? is partition a form of federated? federation consists of partitioning, replication and uniformity. Uniformity not covered federation. properties: autonomy, heterogeneity}

\note{replication could have a value: how many copies? average number of sites
that store a data element; distributedness: average distance to data; does it
make sense to merge partitioning and replicated?}

\note{new properties: coupling and integration (federation is a potential way to do integration).}

\note{federation: loosely coupled, partitioning == loosely coupled? Who is
carrying out the integration of sources? user, application, infrastructure}

\note{data storage vs. data production: Is natural the point where the data
storage is?}

\note{notion of geographic distribution}

\note{This needs to be incorporated in prior subsections: Types of Distributed Data Infrastructure}

\note{volume of data is almost always increasing. Is this not dynamic? We need
to refine the timescale property}

\note{distributed properties: subset (partial replication?) vs. partitioning}

\note{4 could be an entry in every item (except FUSION - since it is always
1 machine). Applies not to special high-end supercomputer. For multi-purpose,
distributed setup infrastructure is an element: data-world analog for a static
execution model. We need to talk about data's use of infrastructure... Would the
data be able to exploit infrastructural changes?
DK: costs for data storage to high to actually do that
}

\katznote{Fusion is not one machine.  4 doesn't apply to a number of applications - it only applies to the applications which have been designed to do this}

\note{view point: application consumer of data? We need to clarify the
viewpoint for each application.}

\note{need to define relationship between production and consumption of data}

\note{not all apps are input-proc-post-proc; there are also feedback systems...}

\note{Use of ``natural'' needs to be defined more carefully. Table needs to be re-visited. }

\note{How to handle feedback to sensors?}


